% Created 2014-11-26 三 21:31
\documentclass{article}
\usepackage[slantfont, boldfont]{xeCJK}
\usepackage{amsmath}
\usepackage{xunicode}
\usepackage{indentfirst}
\usepackage{fontspec}
\usepackage{listings}
\usepackage{xcolor}
\usepackage{hyperref}
\setCJKmainfont{SimSun} % 设置缺省中文字体
\parindent 2em
 
\setmainfont{DejaVu Sans} % 英文衬线字体
\setsansfont{DejaVu Serif} % 英文无衬线字体
\setmonofont{DejaVu Sans Mono} % 英文等宽字体
%\punctstyle{DejaVu Sans} % 开明式标点格式
 
 
\defaultfontfeatures{Mapping=tex-text} %如果没有它,会有一些 tex 特殊字符无法正常使用,比如连字符。
 
% 中文断行
\XeTeXlinebreaklocale "zh"
\XeTeXlinebreakskip = 0pt plus 1pt minus 0.1pt

% 代码设置
\lstset{numbers=left, 
numberstyle= \tiny, 
keywordstyle= \color{ blue!70},commentstyle=\color{red!50!green!50!blue!50}, 
frame=shadowbox, 
rulesepcolor= \color{ red!20!green!20!blue!20} 
} 
\author{w0mTea}
\date{\today}
\title{An Emacs Tutorial for Vim User}
\hypersetup{
 pdfkeywords={},
  pdfsubject={},
  pdfcreator={Emacs 24.3.1 (Org mode 8.3beta)}}
\begin{document}

\maketitle
\tableofcontents

\newpage

\section{前言}
\label{sec-1}
写这篇教程的起因在于向许多vimer推荐org-mode时,
他们总是觉得虽然org-mode功能强大,可是使用emacs总是有着一些障碍。
作为一个同样从vim转向emacs的人,
我觉得或许分享我的经验可以让他们更快的接受emacs,从而体验emacs的美妙。

本文默认读者所使用的是类unix操作系统,
因此对于windows用户来说,如果某些操作在你的电脑上无法执行,请不必大惊小怪。
\section{为什么离开vim改用emacs}
\label{sec-2}
在绝大多数unix/linux教程都教我们用vi/vim的情况下,
许多人从开始就习惯了vim的一切:它的简洁、高效,它的模式,它简单直接的按键绑定……
于是就这样,一批批的vim用户诞生。
然而对于一直和vim并称的emacs,我们往往只是知道它很强,却不清楚它到底强在哪里。
如果你想清楚的了解这个vim的老对手到底凭什么能和vim分庭抗礼,
那么,你可以试着使用它。

如果你是一个lisp爱好者,那么你绝对不能错过emacs。
或许经过漫长的配置,vim同样可以很好的支持lisp,
可是在emacs上,lisp天然就可以被良好的支持。
同时,emacs也可以被lisp扩展。
因此,你写的每一行lisp代码,
或许都可以让你的emacs变的更好用。
写lisp,也是让我离开emacs转向vim的契机。

如果你厌倦了unix哲学,厌倦了做一件事只能用无数的小工具来组合;
或者你喜欢一个大而全的东西,可以帮你做各种各样的事情,
那么emacs无疑是一个更好的选择。
事实上,往往emacs用的越久,每天对着它的时间也越长,
它帮你做的事情也越超出简单的文本编辑。
\section{emacs的安装}
\label{sec-3}
emacs并不像vim/vi那样几乎被所有类unix系统内置,
因此我们往往需要手动安装emacs。

对于有包管理器的系统,使用包管理器通常都可以成功安装emacs。
需要注意的是,某些发行版的仓库默认不安装emacs的GUI版本,
因此需要手动安装emacs-x11或类似名字的包。

需要注意的是,emacs的GUI版本并不是xemacs。
我们使用的emacs实际上是gnu emacs这个分支,
而xemacs则是另一个emacs的分支。
虽然xemacs和gnu emacs有着相当的兼容性,
可是在某些时候难免会碰到奇怪的问题。

对于没有包管理器或仓库没有emacs的系统,可以从
\url{http://ftp.gnu.org/gnu/emacs/}
下载。
\section{基础知识}
\label{sec-4}
毫无疑问,对于一个从未是用过vi/vim的人来说,
使用emacs并不是一个特别让人困惑的事,
最起码开始不是,因为emacs和他之前用过的任何文本编辑器(比如windows下著名的notepad)
从表面上看并没有太大的不同。
然而对于习惯了vi/vim模式操作的人来说,这样的操作模式真是让人无比烦恼:
如果不使用一系列快捷键,效率就会变的无比低下;
可是如果使用快捷键……那都是什么鬼东西!

为了让我们可以更快的接受emacs,我们需要了解一些最最基础的东西。
这些知识并不繁琐,但却总是有很大的帮助
\subsection{快捷键的约定}
\label{sec-4-1}
由于没有vim那样的模式之分,emacs的快捷键总是需要使用组合键。
可是网上查到的C-n, C-x C-s都是啥意思呢?

emacs中,快捷鍵的表示都遵循了一些固有的约定。
C-x表示同時按着ctrl和x,
C-x C-s表示先按ctrl+x,然后按ctrl+s。
当然,也可以按着ctrl不放,然后依次按x和s咯。

同样的,还有M-系列的快捷键。
M-x表示同时按alt和x(alt在不同键盘上可能不同,有可能也叫meta之类的)。
在这样的约定里,还有一些其他的特殊键,比如ESC、RET(回车)等
\subsection{常用快捷键}
\label{sec-4-2}
这里列举一些最最简单同时也最最常用的快捷键。
\begin{center}
\begin{tabular}{|c|c|c|}
\hline
快捷键 & 功能\\
\hline
C-x C-f & 打开某文件\\
C-g & 取消正在输入的命令\\
C-x C-c & 关闭emacs\\
C-x C-s & 保存当前文件\\
M-x & 运行命令\\
\hline
\end{tabular}
\end{center}
\section{简单配置}
\label{sec-5}
emacs的配置文件可以是下列三个中的一个:
\begin{itemize}
\item \textasciitilde{}/.emacs
\item \textasciitilde{}/.emacs.el
\item \textasciitilde{}/.emacs.d/init.el
\end{itemize}

虽然说使用哪一个配置文件都可以,可是我还是建议使用最后一种。
因为这种方案下,你可以把emacs相关的所有配置都放在.emacs.d这个文件夹下,
而不是零散的东堆西散。
尤其在你的配置文件变的很大的时候,你可以轻松的把一些部分拆分成单独的模块放在这个文件夹里。

另外,下文提到的\hyperref[package-management]{包管理}中,最好也把其相关文件放在.emacs.d文件夹下。

至于具体的配置,可以根据自己的需求来弄。\hyperref[documents]{后文}会提供一些好的站点帮助大家完成自己的配置文档。
而一些简单的配置,会在后面的内容里提到。
\section{过渡——evil-mode}
\label{sec-6}
对于刚接触emacs的vimer来说,最难习惯的估计就是光标移动了。
如果还能像vim那样操作无疑会与愉快的多。
而像vim一样操作emacs并不是你一个人的想法,因此也有现成的帮手来帮助我们完成目标,
那就是evil-mode。
\subsection{安装}
\label{sec-6-1}
emacs有着若干种安装扩展的方法,具体的会在下一节讲到。
这里只讲一种我最常用到的也是感觉最方便的方法:el-get安装。

在你的配置文件中加入下列部分:(需要注意的是,el-get的默认位置也在.emacs.d文件夹内)
\begin{verbatim}
(add-to-list 'load-path "~/.emacs.d/el-get/el-get")

(unless (require 'el-get nil t)
  (url-retrieve
   "https://github.com/dimitri/el-get/raw/master/el-get-install.el"
   (lambda (s)
     (end-of-buffer)
     (eval-print-last-sexp))))

;; now either el-get is `require'd already, or have been `load'ed by the
;; el-get installer.

;; now set our own packages
(setq
 my:el-get-packages
 '(el-get                               ; el-get is self-hosting
   switch-window                        ; takes over C-x o
   auto-complete                        ; complete as you type with overlays
   zencoding-mode                       ; http://www.emacswiki.org/emacs/ZenCoding
   color-theme                          ; nice looking emacs
   color-theme-tango))                  ; check out color-theme-solarized

;
;; Some recipes require extra tools to be installed
;;
;; Note: el-get-install requires git, so we know we have at least that.
;;
(when (el-get-executable-find "cvs")
  (add-to-list 'my:el-get-packages 'emacs-goodies-el)) ; the debian addons for emacs

(when (el-get-executable-find "svn")
  (loop for p in '(psvn                 ; M-x svn-status
		   yasnippet            ; powerful snippet mode
		   )
	do (add-to-list 'my:el-get-packages p)))

(setq my:el-get-packages
      (append my:el-get-packages
	      (mapcar #'el-get-source-name el-get-sources)))

;; install new packages and init already installed packages
(el-get 'sync my:el-get-packages)
\end{verbatim}
上述代码段会自动检查是否安装了el-get,并自动在未安装的情况下安装。
注意,这段代码需要系统中安装过git才能运行。同时为了在安装其他扩展时不会出问题,
建议安装svn或cvs。
把上述代码段保存后,重新运行emacs,就会自动安装el-get

el-get安好了,那么怎么安装evil-mode呢?回到上面那段代码,可以看到
\begin{verbatim}
;; now set our own packages
(setq
 my:el-get-packages
 '(el-get                               
   switch-window                        
   auto-complete                        
   zencoding-mode                       
   color-theme                          
   color-theme-tango))
\end{verbatim}
只要在这段代码内添加上我们想要的扩展,而且这个扩展恰好在el-get的仓库内,
那么我们就可以自动的安装并启用对应扩展。
大多数常见扩展都可以被el-get自动找到,evil-mode也不例外。
因此只要在这段代码中加上evil-mode就可以。搞定后和下面的差不多:
\begin{verbatim}
(setq
 my:el-get-packages
 '(el-get                               
   switch-window                        
   auto-complete                        
   evil-mode
   zencoding-mode                       
   color-theme                          
   color-theme-tango))
\end{verbatim}
之后重启emacs,就可以安装了。
\subsection{启用}
\label{sec-6-2}
安装成功后,只需要在配置文件中加入
\begin{verbatim}
(require 'evil)
(evil-mode 1)
\end{verbatim}
就可以全局启用evil-mode。
如果想手动启动evil-mode,把上面的1改成0,
在需要启动的时候按M-x evil-mode RET即可。

现在,vim熟悉的操作,不就回来了吗?
\section{emacs中的包管理}
\label{sec-7}
\label{package-management}
在上一节,我们已经使用了el-get来安装扩展。
只需要在列表中加入你需要的包名就可以自动安装,岂不是爽的很?
这一节会介绍一些el-get的其他用法,以及其他的安装方法。
\subsection{el-get}
\label{sec-7-1}
\subsection{ELPA}
\label{sec-7-2}
\subsection{手动安装}
\label{sec-7-3}
\section{保护你的手指——键位更改}
\label{sec-8}
\section{编程语言配置}
\label{sec-9}
\section{重量级应用——org-mode}
\label{sec-10}
\section{文档和资料}
\label{sec-11}
\label{documents}
\begin{itemize}
\item emacs manual:\url{http://www.gnu.org/software/emacs/manual/html_node/emacs/index.html}
\item emacs wiki: \url{http://www.emacswiki.org/emacs/}
\end{itemize}
\section{结尾}
\label{sec-12}
本文仓促写成,错漏颇多,还望各位指出错误,让这份教程可以帮助更多的人。
% Emacs 24.3.1 (Org mode 8.3beta)
\end{document}
